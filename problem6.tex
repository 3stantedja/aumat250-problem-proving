\documentclass[a4paper,12pt]{article}
\usepackage{problems}
\title{Problem Write-up 6}
\date{2019-10-13}

\begin{document}
    \begin{problem}
        A \textbf{\textit{binary string}} is a “word” in which each “letter” can only be \(0\) or \(1\). Prove that there are \(2^n\) different binary strings of length \(n\).
    \end{problem}
    \begin{answer}
        We want to prove that \(\forall n P(n)\) is true, where \(P(n) = 2^n\), and \(n\) representing the length of a binary string. Each “letter” of the binary string has two possible values, 1 or 0. So let's start with our base case where \(n = 1\).
    	\begin{align}
            P(1) &= 2^1 \notag\\
            &= 2
    	\end{align}
	    As we can see, the base case is true as we've established that for a string of length 1, there are two different strings possible. Now let's assume that the case for \(n = k\) (ie. \(2^k\)) is true. From here we need to prove that the case for \(2^{k + 1}\) is true such that \(\forall k P(k) \rightarrow P(k + 1)\) is true.
    \end{answer}
\end{document}
    