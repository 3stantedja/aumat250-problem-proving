\documentclass[a4paper,12pt]{article}
\usepackage{problems}
\title{Problem Write-up 6}
\date{2019-10-13}

\begin{document}
    \begin{problem}
        A \textbf{\textit{binary string}} is a “word” in which each “letter” can only be \(0\) or \(1\). Prove that there are \(2^n\) different binary strings of length \(n\).
    \end{problem}
    \begin{answer}
        We want to prove that \(\forall n P(n)\) is true, where \(P(n) = 2^n\), and \(n\) representing the length of a binary string. Each “letter” of the binary string has two possible values, 1 or 0. So let's start with our base case where \(n = 1\).
        
        \begin{align} \label{eq:1}
            P(1) &= 2^1 \notag \\
            &= 2
    	\end{align}
       
        As we can see, the base case is true as for a string of length 1, there are two different strings possible (\{0, 1\}). Now let's assume that the case for \(n = k\) (such that the number of different strings are \(2^k\)) is true. From here we need to prove that the case for \(2^{k + 1}\) (where \(n = k + 1\)) is true such that \(\forall k P(k) \rightarrow P(k + 1)\) is true.
        
        The expression \(2^k\) can be broken down as so:
        \begin{equation} \label{eq:2}
            2^k = \underbrace{2 \times 2 \times \dotsb \times 2 \times 2}_{k\ \text{times}}.
        \end{equation}
        Each of them can be considered to be a copy of \(2^1\), that is, the different states of each “letter” as shown in \eqref{eq:1}. As such we know that for the case where \(n = k\), this holds true.

        Now, consider the case where \(n = k + 1\). Then
    \end{answer}
\end{document}
    