\documentclass[a4paper,12pt]{article}
\usepackage{problems,lg_jsylvest}
\title{Problem Write-up 8}
\date{2020-02-12}

\begin{document}
	\begin{problem}
		The equality of sets \[A \cartprod (B \relcmplmnt C) = (A \cartprod B) \relcmplmnt (A \cartprod C)\] is true in general.
		
		Write a \textbf{\textit{formal}} proof of this equality, using the Test for Set Equality.	
	\end{problem}
	\begin{answer}
		Let's start by substituting \(A \cartprod (B \relcmplmnt C)\) as \(M\) and \((A \cartprod B) \relcmplmnt (A \cartprod C)\) as \(N\). To test for the set equality \(M = N\), we need to check that \(M \subseteq N\) and \(N \subseteq M\), applying the Subset Test twice. 

		For the case where \(A \cartprod (B \relcmplmnt C) \subseteq (A \cartprod B) \relcmplmnt (A \cartprod C)\), say that we have an object \(x\) consisting of \((a,b)\) for some \(a\) and \(b\) in \(\mathbb{R}\). Then, \(x \in M\). This implies that \(a \in A\) and \(b \in (B \relcmplmnt C)\). Since we're only getting \(b\) from \(B \relcmplmnt C\), we can say that \(b \in B\).

		Now let's work our way up towards \(N\). We know that \(a \in A\) and \(b \in B\). Doing the Cartesian product \(A \cartprod B\) in \((a,b)\), which is \(x\). Since we're not getting anything from \(A \cartprod C\) due to the relative compliment, we can say that \(x \in N\).

		We can apply this backward in the case where \((A \cartprod B) \relcmplmnt (A \cartprod C) \subseteq A \cartprod (B \relcmplmnt C)\). Assume an object \(x = (a,b)\), applying the same definitions to each \(a\) and \(b\) as specified previously. Then, \(x \in N\). Due to the relative compliment in the statement, we can say that \(x \not\in (A \cartprod C)\), meaning \(x \in (A \cartprod B)\). As such, \(a \in A\), and \(b \in B\).

		Working our way back up to \(M\), the relative compliment in \((B \relcmplmnt C)\) means that we'd pretty much ignore anything from \(C\) or \(B \cap C\). Performing the Cartesian product with \(A\) yields us \(x = (a,b)\), meaning \(x \in M\); hence proving the equality.
	\end{answer}
\end{document}
