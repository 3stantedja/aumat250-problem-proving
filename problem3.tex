\documentclass[a4paper,12pt]{article}
\usepackage{problems}
\title{Problem Write-up 3}
\date{2019-10-01}

\begin{document}
    \begin{problem}
        Let \(P(f,g)\) represent the predicate \(\frac{df}{dx} = g\) where \(f\) and \(g\) are free variables in the domain of \emph{continuous} functions in the real variable \(x\).

        You probably figured out in class that the statement \[(\exists f)(\forall g) P(f,g)\] is \emph{false}. So its negation must be true. Write a careful, thoroughly-convincing argument that this negation is true.

    \end{problem}
    \begin{answer}
        Let's perform the negation of the above statement.
        \begin{align*}
            \neg (\exists f)(\forall g) P(f,g) &\lequiv (\forall f) \neg (\forall g) P(f,g) \\
            &\lequiv (\forall f) (\exists g) \neg P(f,g)
        \end{align*}
        As we can see, the negation of the statement \((\exists f)(\forall g) P(f,g)\) is \((\forall f) (\exists g) \neg P(f,g)\) (by rules of negation of quantifiers). We'd like to argue that the negated statement is true.

        Let there be an arbitrary non-specific continuous function \(f\). Then there's a function \(g\) that is not the derivative of \(f\). We split this into two cases, one where it works for most forms of functions, and the other is when the function \(f\) is \(e^x\).

        
    \end{answer}
\end{document}