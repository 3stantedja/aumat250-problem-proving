\documentclass[12pt,a4paper]{article}
\usepackage{problems,lg_jsylvest}

\title{Problem Write-up 3}
\date{2020-09-02}

\begin{document}
    \begin{problem}
        \emph{Prove by proving the contrapositive:} If \(2^n - 1\) is prime, then \(n\) is prime.
    \end{problem}
    
    \begin{answer}
        First, let's determine the contrapositive of the above statement.
        \begin{indcontrapositive}
        	If \(n\) is not prime, then \(2^n - 1\) is not prime.
        \end{indcontrapositive}
        Now, let's define what a non-prime number \(n\) (now referred as a \emph{composite}) is.
        A composite number \(n\) is when
        \begin{enumerate}
        	\item \(n\) is a positive integer,
        	\item there exists another number greater than 1 and not itself that divides \(n\),
        	\item and \(n > 1\).
    	\end{enumerate}
    	In our contrapositive, since \(n\) is composite, then 
    \end{answer}
\end{document}