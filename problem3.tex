\documentclass[a4paper,12pt]{article}
\usepackage{problems}
\title{Problem Write-up 3}
\date{2019-09-01}

\begin{document}
    \begin{problem}
        Let \(P(f,g)\) represent the predicate \(\frac{df}{dx} = g\) where \(f\) and \(g\) are free variables in the domain of \textit{continuous} functions in the real variable \(x\).

        You probably figured out in class that the statement \[(\exists f)(\forall g) P(f,g)\] is \textit{false}. So its negation must be true. Write a careful, thoroughly-convincing argument that this negation is true.

    \end{problem}
    \begin{answer}
        Let's perform the negation of the above statement.
        \begin{align*}
            \neg (\exists f)(\forall g) P(f,g) &\lequiv (\forall f) \neg (\forall g) P(f,g) \\
            &\lequiv (\forall f) (\exists g) \neg P(f,g)
        \end{align*}
        As we can see, the negation of the statement \((\exists f)(\forall g) P(f,g)\) is \((\forall f) (\exists g) \neg P(f,g)\) (by rules of negation of quantifiers). We'd like to argue that the negated statement is true.

        Let's assume an example: say that our function \(f\) is \(x^2 + 4x + 4\). The statement states that there is a function \(p\) such that \(p\) isn't the derivative of \(f\). There are numerous continuous functions \(g\) that isn't a derivative of \(f\), e.g. the function itself, 
    \end{answer}
\end{document}