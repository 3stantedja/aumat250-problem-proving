\documentclass[a4paper,12pt]{article}
\usepackage{problems}
\title{Problem Write-up 3}
\date{2019-10-01}

\begin{document}
    \begin{problem}
        Let \(P(f,g)\) represent the predicate \(\frac{df}{dx} = g\) where \(f\) and \(g\) are free variables in the domain of \emph{continuous} functions in the real variable \(x\).

        You probably figured out in class that the statement \[(\exists f)(\forall g) P(f,g)\] is \emph{false}. So its negation must be true. Write a careful, thoroughly-convincing argument that this negation is true.

    \end{problem}
    \begin{answer}
        Let's perform the negation of the above statement.
        \begin{align*}
            \neg (\exists f)(\forall g) P(f,g) &\lequiv (\forall f) \neg (\forall g) P(f,g) \\
            &\lequiv (\forall f) (\exists g) \neg P(f,g)
        \end{align*}
        As we can see, the negation of the statement \((\exists f)(\forall g) P(f,g)\) is \((\forall f) (\exists g) \neg P(f,g)\) (by rules of negation of quantifiers). We'd like to argue that the negated statement is true.

        Let there be an arbitrary non-specific continuous function \(f\). Then there's a function \(g\) that is not the derivative of \(f\). One way we could ensure that \(g \not= f'\) is by saying that \(g = f\). This works for most cases (in fact for all but one specific form of \(f\)).

        For instance, here's a pair of \(f\) and \(g\) that satisfies the negated statement.
        \begin{align*}
            \text{Assume that } f(x) &= x^2. \\
            \text{Then } g(x) &= f(x) \\
            &= x^2. \\
            &\therefore \neg P(f,g)
        \end{align*}

        The one case that it breaks down is when we do exponential functions (specifically \(f(x) = \exp(x)\)). What we'd like to do instead in this case is to just pick any other functions as \(g\) \emph{but itself}; i.e. we'd want to pick anything for \(g\) but \(f\) itself when \(f(x) = \exp(x)\). This way, we'd be sure that for any functions \(f\), \(g\) will never fulfil the predicate \(P(f,g)\).
    \end{answer}
\end{document}