\documentclass[12pt,a4paper]{article}
\usepackage{problems,lg_jsylvest}

\title{Problem Write-up 2}
\date{2020-08-31}

\begin{document}
    \begin{problem}
        Let \(P(f,g)\) represent the predicate \(\frac{df}{dx} = g\text{,}\) where \(f\) and \(g\) are free variables in the domain of \emph{continuous} functions in the real variable \(x\).

        You probably figured out in class that 
        \begin{equation} \label{eq:1}
        	(\exists f)(\forall g) P(f,g)
        \end{equation}
         is false. So its negation must be true. Write a careful, thoroughly-convincing argument that this negation is true.
    \end{problem}
    
    \begin{answer}
        First let's apply the negation of Statement \ref{eq:1}.
        \begin{align*}
        	\lgcnot (\exists f)(\forall g) P(f,g) &\lgcequiv (\forall f) \lgcnot (\forall g) P(f,g) \\
        	&\lgcequiv (\forall  f) (\exists g) \lgcnot P(f,g)
        \end{align*}
        We would like to argue that the negation of the statement is true. Assume an arbitrary, continuous function \(f\). 
        The negated statement argues that for the function \(f\) there exists a function \(g\) such that \(g \neq f'\). 
        One example that would cover most cases is when we define \(g = f\). 
        This means for most cases \(g\) would not be a derivative of \(f\) given the predicate \(P(f,g)\), save for cases like exponential functions (e.g. \(\exp{(x)}\)), and absolute functions (e.g. \(\abs{x}\); this is because while technically it is differentiable, at \(x = 0\) it does not exist, hence discontinuous; which means that no matter the case the absolute function does not have a continuous derivative function \(g\)). 
        For the case of \(\exp{(x)}\), picking any other functions for \(g\) will fulfill the negated statement.
    \end{answer}
\end{document}