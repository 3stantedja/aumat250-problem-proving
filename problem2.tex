\documentclass[a4paper,12pt]{article}
\usepackage{problems}
\usepackage{lg_jsylvest}
\title{Problem Write-up 2}
\date{2019-09-29}

% Notes for next iteration
% a & b: need to explicitly say exclusive or
% c: clearer explanation; most are unsupported.

\begin{document}
    \begin{problem}
        In this problem, we will justify the equvalence \[p \leftrightarrow q \Leftrightarrow (p \to q) \wedge (q \to p).\]
        So consider the statements \(A = p \leftrightarrow q\) and \(B = (p \to q) \wedge (q \to p)\).

        \begin{enumerate}
                \item Argue if \(A\) is false, then so is \(B\). Do \textit{not} use the proposed equivalence above as part of your argument.
                \item Argue if \(B\) is false, then so is \(A\). Do \textit{not} use the proposed equivalence above as part of your argument.
                \item Explain why the two arguments in (a) and (b), taken together, justify the equivalence \(A \Leftrightarrow B\). Do this \textit{without} making any further arguments about the truth values of \(p\) and \(q\).
        \end{enumerate}
    \end{problem}

    \begin{answer}
        Here, we're asked to prove the equivalence of two statements, which are now represented as \(A\) and \(B\). We'll prove this by considering the forward case (\(A \Rightarrow B\)) then the backward case (\( A \Leftarrow B \)) in order to justify the equivalence.

        Assume that \(A\) is false. We want to argue that \(B\) is also false. In order for statement \(A\) to be false, either \(p\) or \(q\) need to be false. Now, in order for \(B\) to be false, either \(p \to q\) or \(q \to p\) has to be false; however both canot be false, as having both \(p \to q\) and \(q \to p\) will result in \(B\) being true, which is what not we want. As either \(p\) or \(q\) is false, one of the substatements in \(B\) will be false, and as a result statement \(B\) is false.

        In the case that \(B\) is false, we can also apply the same line of thinking to argue that \(A\) is also false. Let's assume that \(B\) is false. This means that one of the substatements need to be false (because as previously discussed having both substatements true goes against our assumptions), so either \(p\) or \(q\) has to be false. As a result, statement \(A\) will automatically be false because either \(p\) or \(q\) is false, making the biconditional false as well.

        In effect, when we put the two cases together, we could argue that the equvalence \(A \lgcequiv B\) is true. This is because from the arguments above, when \(A\) is false then \(B\) is also false, and vice versa, meaning that the statement \(A \leftrightarrow B\) is tautological as there no cases where \(A\) is true and \(B\) is false, and vice versa (as shown above). As such, the two arguments justifies the equivalence \(A \lgcequiv B\).
    \end{answer}
\end{document}