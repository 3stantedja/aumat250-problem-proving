\documentclass[12pt,a4paper]{article}
\usepackage{problems,lg_jsylvest}

\title{Problem Write-up 1}
\date{2020-08-31}

\begin{document}
	\section{Problem 1}
    \begin{problem}
        Consider the logical statement
        \begin{equation} \label{eq:1.1}
        	(p \lgccond q) \lgccond (\lgcnot p \lgcor q) \text{.}
        \end{equation}
        Argue convincingly that this symbolic statement is a tautology, not by writing out its truth table, but by arguing that it is not possible for the statement to be false.
    \end{problem}
    
    \begin{answer}
        Assume that Statement \ref{eq:1.1} is false. Hence, \(\lgcnot p \lgcor q\) has to be false, and \(p \lgccond q\) is true. This means that \(p\) is true and \(q\) is false.

       	With these truth values for \(p\) and \(q\), \(p \lgccond q\) is false, contradicting the fact that \(p \lgccond q\) has to be true, which means that Statement \ref{eq:1.1} can't be false. As such, \eqref{eq:1.1} is a tautology.
        
    \end{answer}

    \section{Problem 2}
    \begin{problem}
        In this activity, we will justify the equivalence
        \begin{equation} \label{eq:2.1}
        	p \lgcbicond q \lgcequiv (p\lgccond q) \lgcand (q \lgccond p) \text{.}
        \end{equation}
        So consider the statements \(A = p \lgcbicond q\) and \(B = (p\lgccond q) \lgcand (q \lgccond p)\text{.}\)

        Explain why the the two arguments in Task a and Task b, taken together, justify the equivalence \(A \lgcequiv B\text{.}\) Do this without making any further arguments about the truth values of \(p\) and \(q\).
    \end{problem}
    
    \begin{answer}
        Insert answer here
    \end{answer}
    \end{document}