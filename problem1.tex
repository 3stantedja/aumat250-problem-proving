\documentclass[a4paper,12pt]{article}
\usepackage{problems}
\title{Problem Write-up 1}
\date{2019-09-25}

\begin{document}
\begin{problem}
    Proof that the logical statement \[(p \to q) \to (\neg p \vee q)\] is tautological, by arguing that it is not possible for the statement to be false.
\end{problem}

\begin{answer}
   A logical statement is said to be a tautology if the statement in itself is always true. In this case, we need to prove that the statement above is tautological by showing that it's not possible for the logical statement above be false.

   Suppose that the statement above is false. This means that the first substatement (\(p \to q\)) needs to be true and the second substatement (\(\neg p \vee q\)) be false (per the conditional truth table). In order for the second substatement to be false, the value for the value of p needs to be true, and the value of q be false (per the disjunction truth table). Applying these values to the first substatement causes it to be false (per the disjunction truth table). However, earlier on we mentioned that the first substatement needs to be true! Because the first substatement is false, the overall logical statement is true. As such it is not possible for this statement to be false.
\end{answer}
\end{document}