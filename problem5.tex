\documentclass[a4paper,12pt]{article}
\usepackage{problems}
\title{Problem Write-up 5}
\date{2019-10-09}

\begin{document}
	\begin{problem}
		Prove by proving the contrapositive: \textit{If \(2^n - 1\) is prime, then \(n\) is prime}.
	\end{problem}
	\begin{answer}
		Let's begin by stating the contrapositive of the above statement.
		\begin{contrapositive}
			If \(n\) is not prime, then \(2^n - 1\) is not prime.
		\end{contrapositive}
		If a number is not prime, therefore it would be composite. Assuming the number \(n\) is a positive integrer, this means that \(n\) would be divisble by any number other than 1 or itself. In our case, we shall ignore the possibility for when \(n\) is 1.
	\end{answer}
\end{document}