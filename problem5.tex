\documentclass[a4paper,12pt]{article}
\usepackage{problems}

\addbibresource{bibliography/problem5.bib}

\title{Problem Write-up 5}
\date{2019-10-09}

\begin{document}
	\begin{problem}
		Prove by proving the contrapositive: \textit{If \(2^n - 1\) is prime, then \(n\) is prime}.
	\end{problem}
	\begin{answer}
		Let's begin by stating the contrapositive of the above statement.
		\begin{indcontrapositive}
			If \(n\) is not prime, then \(2^n - 1\) is not prime.
		\end{indcontrapositive}
		This means that \(n\) could be composite, or 1, due to the fact that 1 is neither prime nor composite. As \(n = 1\) is a trivial case, we shall not prove it here.
		
		In order for \(n\) be composite, it needs to be
		\begin{enumerate}[1.]
			\item a positive integer,
			\item greater than 1, and
			\item there exists a number greater than 1 and not itself that divides it.
		\end{enumerate}
		Now, assume that \(n\) is composite. There exists two numbers \(x,y > 1\) such that 
		\begin{equation} \label{eq:1}
			n = xy
		\end{equation}
		where \(x\), \(y\) are the factors of \(n\). From here, we can use this fact to prove that \(2^n - 1\) is composite. Substitute in \eqref{eq:1} such that
		\begin{align} \label{eq:2}
			2^n - 1 &= 2^{xy} - 1 \nonumber \\
			&= (2^x)^y - 1.
		\end{align}
		Using the difference of power factorisation formula, we factorise \eqref{eq:2} like so:
		\begin{equation}
			(2^x)^y - 1 = (2^x - 1)((2^x)^{y - 1} + (2^x)^{y - 2} + \dotsb + (2^x)^2 + (2x)^1 + 1)
		\end{equation}
		As both \(x,y > 1\), this means that \(x,y < n\), which means \[(2^x - 1),((2^x)^{y - 1} + (2^x)^{y - 2} + \dotsb + (2^x)^2 + (2x)^1 + 1)) > 1\] and \[(2^x - 1),((2^x)^{y - 1} + (2^x)^{y - 2} + \dotsb + (2^x)^2 + (2x)^1 + 1) < 2^n - 1.\] Because \(2^x - 1\) is a proper divisor (and so is the second divisor), therefore \(2^n - 1\) is composite, proving the contrapositive.
	\end{answer}
	\section*{Acknowledgement}
	This solution is based on a write-up on Mathematics Stack Exchange.\cite{496418} Effort has been put in to avoid plagiarising the source material while maintaining originality in exposition and working.
	
	\printbibliography
\end{document}