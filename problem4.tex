\documentclass[a4paper,12pt]{article}
\usepackage{problems}
\title{Problem Write-up 4}
\date{2019-10-08}

\begin{document}
    \begin{problem}
        Show that the following argument is valid \textit{without} using a truth table. Instead, argue that the argument fulfills the equivalent definition for \textbf{\textit{valid argument}} that you created in Activity 5.4.
        \[
            \begin{array}{l}
                p \rightarrow \neg q \\
                r \rightarrow (p \wedge q) \\ \hline
                \neg r
            \end{array}
        \]
    \end{problem}
    \begin{answer}
        The definition of a valid agrument is:

        \begin{displayquote}
            Whenever the premises are all true, the conclusion is true as well.
        \end{displayquote}

        Activity 5.4 had us produce the contrapositive of the above definition, which is:

        \begin{displayquote}
            Whenever the conclusion is false, one of the premises are false.
        \end{displayquote}
        
        We'll use this to argue that the above argument is valid.

        Assume that the conclusion is false. This means that the value of r is true. Now, let's assume that the premise \(r \rightarrow (p \wedge q)\) is true. In order for it to be true, both \(p\) and \(q\) must be true (per the conditional and the conjunction truth table). But, this means that the premise \(p \rightarrow \neg q\) is false (per the conditional truth table). As such, we have showed that this argument is valid as the equivalent statement requires that one of the premises must be false for the conclusion to be false.
    \end{answer}
\end{document}
